{\bfseries Authors\+:}~\newline
 ~~~~{\itshape  Matthijs Daggelders }~\newline
 ~~~~{\itshape  Niek Ratering Arntz }~\newline
\hypertarget{index_intro_sec}{}\section{Introduction}\label{index_intro_sec}
~~~~This application runs on a S\+T\+M32\+F4-\/discovery board and drives a 320x240 pixel V\+GA screen.~\newline
 ~~~~Different inputs can be given to the application using a U\+A\+RT connection. With the input thats~\newline
 ~~~~given a variety of shapes, text and bitmaps can be displayed on the screen. It\textquotesingle{}s also possible~\newline
 ~~~~to buffer the given inputs and repeat them.\hypertarget{index_usage_sec}{}\section{Usasage}\label{index_usage_sec}
~~~~The program runs via a U\+A\+RT connection on pins P\+A2(\+Tx) and P\+A3(\+Rx) and drives a 320x240 pixel~\newline
 ~~~~V\+GA screen.\+It uses the \href{}{\tt E\+E-\/\+A\+P\+I.\+lib} to drive the V\+GA screen and the U\+A\+RT connection.~\newline
\hypertarget{index_uart_sec}{}\subsection{U\+A\+R\+T connection}\label{index_uart_sec}
~~~~-\/ Baud\+Rate = 115200 baud~\newline
 ~~~~-\/ Word Length = 8 Bits~\newline
 ~~~~-\/ One Stop Bit~\newline
 ~~~~-\/ No parity~\newline
 ~~~~-\/ Hardware flow control disabled (R\+TS and C\+TS signals)~\newline
 ~~~~-\/ Receive and transmit enabled~\newline
 ~~~~-\/ Pins P\+A2(\+Tx) and P\+A3(\+Rx)~\newline
 ~~~~-\/ tested throughput 10 commands/s~\newline
\hypertarget{index_input_sec}{}\subsection{possible commands}\label{index_input_sec}
~~~~line = {\itshape \char`\"{}lijn,x,y,x2,y2,width,color\char`\"{}}~\newline
 ~~~~ellipse = {\itshape \char`\"{}ellips,x-\/mp,y-\/mp,radius-\/x,radius-\/y,color,fill\char`\"{}}~\newline
 ~~~~rectangle = {\itshape \char`\"{}rechthoek,x-\/lo,y-\/lo,x-\/rb,y-\/rb,color,fill\char`\"{}}~\newline
 ~~~~traingle = {\itshape \char`\"{}driehoek,x,y,x2,2,x3,y3,color,fill\char`\"{}}~\newline
 ~~~~text = {\itshape \char`\"{}tekst,x,y,text,fontname,color,style\char`\"{}} \mbox{[}style\+: \char`\"{}normaal\char`\"{},\char`\"{}vet\char`\"{},\char`\"{}cursief\char`\"{}\mbox{]}~\newline
 ~~~~bitmap = {\itshape \char`\"{}bitmap,nr,x-\/lo,y-\/lo\char`\"{}} \mbox{[}nr; 0=happysmily, 1=sadsmiley, 2=arrow-\/up, 3=arrow-\/down,~\newline
 ~~~~~~~~~~~~~~~~~~~~~~~~~~~~~~~~4=arrow-\/left, 5=arrow-\/right\mbox{]}~\newline
 ~~~~clearscreen = {\itshape \char`\"{}clearscherm,color\char`\"{}}~\newline
 ~~~~wait = {\itshape \char`\"{}wacht,ms\char`\"{}}~\newline
 ~~~~repeat = {\itshape \char`\"{}repeat\char`\"{}}~\newline


$\ast$\+The idle line detection of the U\+A\+RT isn\textquotesingle{}t used so all commands should end with an carriage return.\hypertarget{index_colors_sec}{}\subsection{possible colors}\label{index_colors_sec}
~~~~The application has 16 colors where the user can choose from.~\newline
 ~~~~the colors are\+:~\newline
 ~~~~black = \char`\"{}zwart\char`\"{}~\newline
 ~~~~blue = \char`\"{}blauw\char`\"{}~\newline
 ~~~~light blue = \char`\"{}lichtblauw\char`\"{}~\newline
 ~~~~green = \char`\"{}groen\char`\"{}~\newline
 ~~~~lightgreen = \char`\"{}lichtgroen\char`\"{}~\newline
 ~~~~cyan = \char`\"{}cyaan\char`\"{}~\newline
 ~~~~light cyan = \char`\"{}lichtcyaan\char`\"{}~\newline
 ~~~~red = \char`\"{}rood\char`\"{}~\newline
 ~~~~light red = \char`\"{}lichtrood\char`\"{}~\newline
 ~~~~magenta = \char`\"{}magenta\char`\"{}~\newline
 ~~~~light magenta = \char`\"{}lichtmagenta\char`\"{}~\newline
 ~~~~brown = \char`\"{}bruin\char`\"{}~\newline
 ~~~~yellow = \char`\"{}geel\char`\"{}~\newline
 ~~~~gray =\char`\"{}grijs\char`\"{}~\newline
 ~~~~white =\char`\"{}wit\char`\"{}~\newline
\hypertarget{index_repeat_sec}{}\subsection{Repeat function.}\label{index_repeat_sec}
~~~~To use the repeat function the input commands need to be buffered. To achieve this the program ~\newline
 ~~~~needs to get a wait(\char`\"{}wacht\char`\"{}) command first, the commands after the wait will be buffered. If the ~\newline
 ~~~~last command in the buffer is \char`\"{}repeat\char`\"{} everything including the first wait will be repeated.~\newline
 ~~~~To exit the loop send a single command and the loops ends at the end of the buffer and resets.~\newline
~\newline
 ~~~~{\bfseries N\+O\+TE\+:} giving the repeat command while the buffer is empty the program crashes. So make sure you have enough wait time.~\newline
 